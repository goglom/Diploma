\chapter{Библиография}\label{ch:bib}



В данном шаблоне используется подход к генерированию библиографии с помощью bibtex. Эта система позволяет генерировать список литературы в заданном стиле используя базу данных статей в определённом формате.

\begin{itemize}
	\item \href{https://ru.wikipedia.org/wiki/BibTeX}{https://ru.wikipedia.org/wiki/BibTeX}
	\item  \href{https://www.overleaf.com/learn/latex/Bibliography_management_with_bibtex}{\small https://www.overleaf.com/learn/latex/Bibliography\_management\_with\_bibtex}
	\item \href{http://mydebianblog.blogspot.com/2006/11/latex-jabref.html}{http://mydebianblog.blogspot.com/2006/11/latex-jabref.html}
	\item 	\href{http://www.bibtex.org/Using/}{http://www.bibtex.org/Using/}
	\item \href{https://en.wikibooks.org/wiki/LaTeX/Bibliography_Management}{https://en.wikibooks.org/wiki/LaTeX/Bibliography\_Management}
\end{itemize}

\section{Использование bibtex}

\begin{enumerate}
	\item Либо в преамбуле документа, либо непосредственно перед вызовом команды построения библиографии указать её стиль:  
	
	\verb|\bibliographystyle{Style}|
	
	Здесь \verb|Style| -- стиль. Есть предустановленные:
	
	\href{https://www.overleaf.com/learn/latex/bibtex_bibliography_styles}{https://www.overleaf.com/learn/latex/bibtex\_bibliography\_styles}
	
	Но можно использовать сторонние, указав путь к соответствующему \verb|.bst| файлу.
	\item В тексте ссылка на литературу указывается с помощью команды: 
	
	\verb|\cite{PubName}| (и её вариаций), получается: \cite{Annenkov2018,Annenkov2019,Annenkov2020,AnnenkovThesis2019}
	
	\verb|PubName| -- идентификатор статьи, хранящейся в базе данных статей (простой текстовый файл \verb|.bib|)
	
	\item В том месте где мы хотим сгенерировать библиографию вызываем команду:
	
	\verb|\bibliography{RefSource}|
	
	Здесь \verb|RefSource| -- адрес  \verb|.bib| файла с базой литературы. Записи в нём хранятся в виде:
	
\begin{small}
\begin{verbatim}
	@article{Annenkov2020,
		author = {Annenkov, V. V. and Volchok, E. P. and 
			Timofeev, I. V.},
		doi = {10.3847/1538-4357/abbef2},
		issn = {15384357},
		journal = {The Astrophysical Journal},
		month = {nov},
		number = {2},
		pages = {88},
		title = {{Electromagnetic Emission Produced by 
				Three-wave Interactions in a Plasma with 
				Continuously Injected Counterstreaming Electron Beams}},
		url = {https://iopscience.iop.org/article/10.3847/
			1538-4357/abbef2},
		volume = {904},
		year = {2020}
	}
	
\end{verbatim}

\end{small}

Здесь \verb|Annenkov2020|~--- идентификатор публикации. Как правило, с сайтов журналов можно брать информацию о статьях сразу в таком формате.

\item  Последовательно применить к документу команды:

\begin{verbatim}
latex doc.tex
latex doc.tex
bibtex doc.tex
latex doc.tex
\end{verbatim}

Здесь вместо \verb|latex| может быть любая другая команда обработки \verb|.tex| файла (\verb|pdflatex, xelatex ....|). 

Зачастую IDE сами по нажатию одной кнопки проделывают всё это.
\end{enumerate}

\section{Поддержка базы данных статей}

Базу данных публикаций в формате bibtex'а можно наполнять и поддерживать с помощью разных инструментов:

\begin{enumerate}
	\item JabRef \href{https://www.jabref.org/}{https://www.jabref.org/} 
	\item Zotero \href{https://www.zotero.org/}{https://www.zotero.org/}
	\item Mendeley \href{https://www.mendeley.com/}{https://www.mendeley.com/}
\end{enumerate}

Эти системы позволяют в удобной форме хранить и обрабатывать информацию о публикациях. Импортировать её с сайтов, pdf файлов статей и т.п.

Последние две системы также хранят всю базу в облаке.


\section{Ссылки на русскоязычные источники}

Для этого при использовании \verb*|\bibliographystyle{ugost2008}| стиля библиографии  (по-умолчанию в данном шаблоне) необходимо в соответствующей записи в .bib файле явно указать язык: \verb*|language={russian}|, например:

\begin{verbatim}
@phdthesis{AnnenkovThesis2019,
	author = {Анненков, В. В.},
	pages = {1--105},
	language={russian},
	school = {ИЯФ СО РАН},
	title = {{ЭЛЕКТРОМАГНИТНАЯ ЭМИССИЯ В ТОНКОЙ
			 ПУЧКОВО-ПЛАЗМЕННОЙ СИСТЕМЕ}},
	url = {https://inp.nsk.su/images/diss/Annenkov_disser.pdf},
	year = {2019}
}
\end{verbatim}



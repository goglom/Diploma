\chapter{Таблицы}\label{ch:tab}
\begin{itemize}
	\item 	\href{http://mydebianblog.blogspot.com/2009/01/tables-in-latex.html}{http://mydebianblog.blogspot.com/2009/01/tables-in-latex.html}
	
	\item 	\href{http://mydebianblog.blogspot.com/2013/01/advanced-tables-in-latex.html}{http://mydebianblog.blogspot.com/2013/01/advanced-tables-in-latex.html} 
\end{itemize}

Для некоторых приведённых примеров требуются пакеты:

окружение таблиц tabularx:\verb|\usepackage{tabularx}|

объединение строк в таблицах:
\verb|\usepackage{multirow}|

для всяких украшательств в таблицах:
\verb|\usepackage{booktabs}|

работа с ``плавающими'' объектами:
\verb|\usepackage{float}|


По ГОСТу оформлены таблицы \ref{tab:GOST1} и \ref{tab:GOST2}.


\begin{table}[ht]
	%\renewcommand{\arraystretch}{1.8} %% increase table row spacing
	%\renewcommand{\tabcolsep}{1cm}   %% increase table column spacing
	\caption{Простая таблица}
	\begin{tabular}{|c|c|r|l|}
		\hline
		раз & два & три & четыре \\
		\hline
		1 & 2 & 3 & 4 \\
		\hline
		пять & шесть & семь & восемь \\
		\hline
		9 & 10 & 11 & 12 \\
		\hline
	\end{tabular}
\end{table}


\begin{table}[H]
		\caption{Такая таблица по ГОСТу}
		\label{tab:GOST1}
	\begin{center}
		\begin{tabular}{|c|c|c|}
			\hline
			\multirow{3}{*}{Размеры нестандартных болтов} & \multicolumn{2}{c|}{Диаметр} \\
			\cline{2-3}
			& Норма & Разброс \\
			\cline{2-3}
			& 10 мм & 1 мм \\
			\hline
		\end{tabular}
	\end{center}
\end{table}

\begin{table}[H]
		\caption{Такая таблица по ГОСТу}
		\label{tab:GOST2}
	\begin{center}
		\begin{tabular}{|c|c|c|}
			\hline
			& \multicolumn{2}{c|}{Диаметр} \\
			\cline{2-3}
			\raisebox{1.5ex}[0cm][0cm]{Нестандартные болты}
			& Норма & Разброс \\
			\hline
			Размеры & 10 мм & 1 мм \\
			\hline
		\end{tabular}
	\end{center}
\end{table}

\begin{table}[htbp]
	\caption{Таблица по ширине страницы}
	\begin{tabularx}{\textwidth}{XXXXX}
		\toprule
		auto & break & case & char & const\\
		%	\hline
		continue   & default   & do & double & else\\
		%	\hline
		\bottomrule
	\end{tabularx}
\end{table}

\begin{table}[htbp]
	\centering
	\caption{Изменённое соотношение ширин столбцов}
	\begin{tabularx}{.85\textwidth}{>{\hsize=0.075\textwidth}XX>{\hsize=0.075\textwidth}XX}
		\toprule
		== & Равно & != & Не равно\\
		> & Больше & < & Меньше\\
		\bottomrule
	\end{tabularx}
\end{table}


\begin{table}[!hbt]
		\caption{Таблица с фиксированной шириной колонок и увеличенным размером строк.}
	\begin{center}
		\renewcommand{\arraystretch}{1.2} %% increase table row spacing	
		\begin{tabular}{|>{\raggedright\arraybackslash}m{3.2cm}|>{\arraybackslash}p{1.9cm}|>{\raggedright\arraybackslash}m{5cm}|>{\raggedright\arraybackslash}m{5cm}|}
			\hline
			\textbf{Величина} & \textbf{Обоз-е} & \textbf{Значение в СИ} & \textbf{Значение в СГС} \\
			\hline
			\textbf{Масса электрона} & $m_e$ &$9.1094\cdot10^{-31}$, кг  & $9.1094\cdot10^{-28}$, г  \\
			\hline
			
		\end{tabular}
	\end{center}
\end{table}	
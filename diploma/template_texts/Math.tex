\chapter{Формулы}\label{ch:eq}
\begin{itemize}
	\item \href{http://mydebianblog.blogspot.com/2009/01/latex-math-in-latex.html}{http://mydebianblog.blogspot.com/2009/01/latex-math-in-latex.html}
\item 	\href{https://www.overleaf.com/learn/latex/mathematical_expressions}{https://www.overleaf.com/learn/latex/mathematical\_expressions}
\end{itemize}

\section{включённые в текст}
Без ограничения общности будем считать, что свет распространяется вдоль координаты $r$. Тогда элемент метрики есть просто $dR = dr/\sqrt{1 - kr^2}$, где $k = 0, +1, -1$ для пространства с нулевой, положительной или отрицательной кривизной, соответственно. Пусть свет был испущен в точке с координатой $r_{em}$ в момент времени $t_{em}$ и принят в точке с координатой $r_{obs} = 0$ в момент времени $t_{obs}$.
\section{вне текста}
Темп  расширения  Вселенной,  т. е.  относительное  увеличение  расстояний  в  единицу  времени,  характеризуется  параметром  Хаббла $$H(t)\equiv\dfrac{\dot{a}(t)}{a(t)}.$$ Параметр  Хаббла  зависит  от  времени;  для  его  современного  значения  применяем, как  обычно,  обозначение  $H_0$.

\begin{gather}\label{f1}
	z=\dfrac{\lambda_{obs}-\lambda_{em}}{\lambda_{em}}\\
	H(t)\equiv\dfrac{\dot{a}(t)}{a(t)}.
\end{gather}

Без нумерации (добавить *)

\begin{gather*}\label{f2}
	z=\dfrac{\lambda_{obs}-\lambda_{em}}{\lambda_{em}}\\
	H(t)\equiv\dfrac{\dot{a}(t)}{a(t)}.
\end{gather*}


С выравниванием (относительно позиции обозначенной \&)
\begin{align}\label{f3}
	\dfrac{\partial\rho}{\partial t}+\nabla(\rho\mathbf{v})&= 0\\
	\dfrac{\partial\mathbf{v}}{\partial t}+(\mathbf{v}\nabla)\mathbf{v}&=-\dfrac{1}{\rho}\nabla P-\nabla\psi\\
	\dfrac{d}{dt}\left(\dfrac{P}{\rho^\gamma}\right)&=0\\
	\nabla^2\psi&=4\pi G\rho
\end{align}

Ссылки на формулы: \ref{f1} и \ref{f2}

Сайты, позволяющие набирать формулы и сразу конвертировать их в изображения (растровые и pdf) (полезно для последующего импорта в PowerPoint презентации):
\begin{itemize}
	\item \href{https://latex.codecogs.com/eqneditor/editor.php}{https://latex.codecogs.com/eqneditor/editor.php} 
	\item \href{https://www.latex4technics.com/}{https://www.latex4technics.com/}
\end{itemize}

Сайт, конвертирующий рукописный ввод в \LaTeX~код:

\href{https://webdemo.myscript.com/views/math/index.html}{https://webdemo.myscript.com/views/math/index.html}
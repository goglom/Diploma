\chapter{Модуль генерации уравнений для геометрических ограничений}\label{ch:modeler}

Как было описано в главе \ref{ch:geomsolver}, задача удовлетворения геометрических ограничений в итоге сводится к численному решению СНАУ. Но для того, чтобы получить СНАУ необходимо, во-первых, представить элементарные геометрические объекты и тела в виде набора переменных. Телом в данном случае называется сложный геометрический объект, представляющий собой набор элементарных геометрических объектов неподвижных относительно друг друга. Во-вторых, для каждого из ограничений генерируется набор уравнений относительно переменных, соответствующих аргументам ограничения. Таким образом в итоге получается система уравнений, описывающая задачу удовлетворения геометрических ограничений. Далее будет представлен набор поддерживаемых геометрических объектов и ограничений, а также описаны способы представление гомерических объектов и в конце будут описаны подходы и особенности описания ограничений набором алгебраических уравнений.

\section{Обзор поддерживаемых объектов и ограничений}

В рамках данной в модуле генерации уравнений была реализована поддержка геометрических объектов и ограничений, представлены в таблице \ref{tab:supp_obj_and_constr}. Ограничения могут задаваться между представленными объектами, если это имеет смысл. Например, между двумя точками ограничение угла не определено. В дальнейшем, планируется поддержка дополнительных геометрических объектов, например сфера, цилиндр и др. Но Описанные далее способы моделизации объектов и методы генераций уравнений могут быть также применены и к таким объектам.

\begin{table}[htbp]
	\caption{Поддерживаемые геометрические объекты и ограничения}
	\label{tab:supp_obj_and_constr}
	\begin{tabularx}{\textwidth}{XX}
		\toprule
		\textbf{Объекты} & \textbf{Ограничения}\\
		\hline
		точка       & совпадение\\
		прямая      & расстояние\\
		плоскость   & касание   \\
		окружность  & фиксация  \\
		            & угол      \\
		            & параллельность\\
		            & перпендикулярность\\
		% & \\
		\bottomrule
	\end{tabularx}
\end{table}

В разделе \ref{sec:middle_level} были упомянуты параметры геометрических ограничений. В целом для ограничений существуют следующие параметры: \textit{выравнивание}, \textit{ориентация} и \textit{вспомогательная точка}. Параметры выравнивание и ориентация могут принимать три значение: положительное, отрицательное и неопределённое. Параметр выравнивание отвечает за поведение направляющих векторов, когда ограничение требует параллельности направляющих векторов обоих геометрий. Например расстояние между плоскостями требует параллельности их нормалей, и в таком случае положительное выравнивание означает со-направленность нормалей, а отрицательное наоборот. В случае неопределенного выравнивания допустимы оба варианта. Параметр ориентация определён только для ограничения расстояния, хотя бы одним аргументом которого является поверхность, и отвечает он за положение объекта относительно нормали поверхности. Например, для ограничения расстояния между плоскостью и точкой возможны два решения: точка находится спереди плоскости (т.е. нормаль направлена к точке) или же точка находится за плоскостью. В общем случае, положительная ориентация означает, что нормаль указывает на объект, а   с отрицательным значение -- наоборот. В таблице \ref{tab:constr_params} указано, какие параметры могут быть у каждого из ограничений. Также, часть ограничений не имеет параметров. А ограничение фиксации имеет собственный уникальный параметр, который позволяет задать полную, или частичную фиксацию объекта.   

\begin{table}[htbp]
    \renewcommand{\arraystretch}{1.4}
	\caption{не числовые параметры ограничений: В -- выравнивание, О -- ориентация, ВТ -- вспомогательная точка, ТФ -- тип фиксации}
	\label{tab:constr_params}
	\begin{tabularx}{\textwidth}{XX}
		\toprule
		\textbf{Ограничение} & \textbf{Параметры}\\
		\hline
		совпадение         & В, ВТ      \\ \hline
	    расстояние         & В, ВТ   \\ \hline
	    касание            & В, О, ВТ   \\ \hline
	    фиксация           & ТФ         \\ \hline
	    угол               & --         \\ \hline
	    параллельность     & В          \\ \hline
	    перпендикулярность & --         \\
	    \bottomrule
	\end{tabularx}
\end{table}


\section{Способы моделирования геометрических объектов и тел}
Как было упомянуто ранее, в первую очередь для генерации уравнений необходимо задать набор переменных, над которыми и будут далее заданы уравнения. Существует как минимум два подхода к представлению геометрических объектов \cite{ershov2007algorithms}. 

Первый подход основан на представлении искомых координаты элементарных геометрических объектов в виде переменных системы. Например, точку можно представить в виде набора трёх переменных, соответствующих её координатам. Отдельно стоит рассмотреть представление направляющих векторов, т.к. при генерации уравнений очень удобно опираться на единичную длину направляющих векторов. Для гарантии единичной нормы векторов, можно представлять такие вектора в виде двух углов ($\vartheta$ и $\varphi$), как представляется любой вектор в сферической системе координат с единичным радиусом. 

\begin{equation*}
    \begin{matrix}
        & x  = \cos{\varphi} \cdot \sin{\vartheta} \\ 
        & y  = \sin\varphi \cdot \sin\vartheta \\
        & z  = \cos\vartheta
    \end{matrix}
\end{equation*}

Но в таком случае координаты такого вектора выражаются через произведения тригонометрических функции от переменных-улов. И как показывает практика, такой подход обеспечивает более плохую сходимость решателя по сравнению со следующим подходом. В этом случае компоненты вектора являются переменными ($x, y, z$), а для обеспечения единичной нормы такого вектора в систему уравнений добавляется уравнение обеспечивающее единичную норму:  

\begin{equation*}
    x^2 + y^2 + z^2 - 1 = 0
\end{equation*}

Исходя из описанных выше соображений, для каждого из поддерживаемых объектов был выбран следующий способ моделирования -- представления в виде набора переменных:

\begin{itemize}
    \item
        \textbf{Точка} -- координаты самой точки; в итоге \textbf{3} переменных.
    \item
        \textbf{Прямая} -- координаты точки на прямой, направляющий вектор; в итоге \textbf{6} переменных.
    \item
        \textbf{Плоскость} -- расстояние от плоскости до начала координат, нормаль (направляющий вектор); в итоге \textbf{4} переменных.
    \item
        \textbf{Окружность} -- координаты центра, нормаль к плоскости (направляющий вектор), радиус; в итоге \textbf{7} переменных.
\end{itemize}

Вторым походом к моделированию объектов является представление положений объектов в виде трансформаций, описывающих переход положения объекта из исходного в искомое. Трансформация в таком случае представляется в виде поворота и смещения, и её параметры являются переменными, соответствующими геометрическому объекту. Описать трансформацию можно несколькими способами. Первый, и не очень эффективный метод, 12-параметрическое моделирование, когда трансформация представляется виде 9 переменных для матрицы поворота и 3 переменных вектора для смещения:

\begin{equation*}
    R = \begin{pmatrix}
        x_1 & x_2 & x_3\\
        x_4 & x_5 & x_6\\
        x_7 & x_8 & x_9
    \end{pmatrix},
    t = \begin{pmatrix}
        x_{10}\\
        x_{11}\\
        x_{12}
    \end{pmatrix}
\end{equation*}
где, $R$ -- матрица поворота, $t$ -- вектор смещения. Поскольку матрица $R$ должна быть матрицей поворота в трёхмерном пространстве, необходимо чтобы она была ортогональной и её определитель равнялся 1. Для этого необходимо 6 дополнительных уравнений для обеспечения требуемых свойств матрицы $R$.

Другой способ является 6-параметрическим и в нём переменными являются 3 угла Эйлера ($\alpha, \beta, \gamma$), описывающих поворот, а также 3 компоненты вектора смещения ($x, y, z$). Такой подход позволяет избавиться от дополнительных уравнений и и обеспечить меньше количество переменных в системе. Существует 12 возможных описаний матрицы поворота с помощью углов Эйлера \textbf{[ссылка на книгу из статьи на Вики]}, и как показала практика выбор представления матрицы поворота влияет на сходимость численного решения, но это влияние не существенно. Но главной особенностью такого подхода является описание геометрических тел. В таком случае описанная трансформация является общей, для набора геометрических объектов, каждое из которых имеет собственные координаты в локальной системе координат тела. В таком случае координаты любого объекта получаются путём применения трансформации всего тела к локальным координатам объекта. Для описания одного объекта такой подход оказываются не эффективен на практике, т.к. заметно ухудшается сходимость решателя при таком способе моделирования объектов. Поэтому при реализации рассматриваемого модуля был выбран смешанный подход, в котором для моделирования одиночных геометрических объектов используется первый из описанных подходов, а для моделирования геометрических тел используется второй подход. 

\section{Особенности генерируемых уравнений}

Как было описано в разделе \ref{sec:low_level}, для решения СНАУ в разрабатываемом решателе используется метод Ньютона-Рафсона, требующий вычислений якобиана системы уравнений на каждой итерации. Этот факт, приводит к тому, что все генерируемые уравнения должны быть дифференцируемы (т.е. частные производные определены во всей области допустимых значений переменных) по всем переменным, от которых они зависят. В противном случае, появление NaN (Not A Number -- не число) или бесконечных значений в якобиане приводит к некорректному решению СЛАУ и в следствии к расходимости метода Ньютона-Рафсона. Исходя из этого факта, требование дифференцируемости  уравнений на всей области определения существенно и должно учитываться при реализации вычислений. 

Одним из первых мест, где проблемы с дифференцируемостью уравнений возникают проблемы, являются уравнения описывающие расстояние между объектами. Например, расстояние $d_0$ между двумя точками, очевидным образом можно задать уравнением:
\begin{equation}\label{eq:pt_pt_dist_bad}
   \left \|  \Vec{p_1} - \Vec{p_2} \right \|_2 - d_0 = 0
\end{equation}
где: $\Vec{p_1}$, $\Vec{p_2}$ -- координаты первой и второй точки соответственно, $d_0$ -- заданное ограничением расстояние между точками. Но в случае если обе точки совпадают, то все частные производные по компонентам $\Vec{p_1}, \Vec{p_2}$ будут не определены. Такое может происходить в различных случаях. Во-первых, пользователь может задать исходное положение с совпадающими точками (особенно если при построении задачи объекты копируются). Во-вторых, расстояние между точками $d_0$ может быть нулевым. А также, в процессе решения координаты точек могут совпасть. И во всех этих случаях в якобиане системы уравнений появляются NaN значения и метод после этого начинает расходится. Избежать этого, можно с помощью кусочно заданных функций. Так, уравнение \ref{eq:pt_pt_dist_bad} может быть заменено следующим уравнением:

\begin{equation*}\label{eq:pt_pt_dist_good}
\left\{\begin{matrix}
 \left \|  \Vec{p_1} - \Vec{p_2} \right \|_2 - d_0 = 0, 
 & \text{ если }  \left \|  \Vec{p_1} - \Vec{p_2} \right \|_2 \leq \veps \\ 
 
 \left \|  \Vec{p_1} - \Vec{p_2} \right \|_2^2 - d_0^2 = 0, 
 & \text{ иначе }
  
\end{matrix}\right.    
\end{equation*}
где: $\veps$ -- допустимая погрешность решения. Можно заменить это кусочно заданное уравнение на второе, не потеряв свойство дифференцируемости, однако, на практике с кусочно заданной функцией решатель сходится в лучше. Подобный подход, также используется при описание других ограничений расстояния, а именно там, где в уравнении вычисляется евклидова норма вектора. Например, это используется в расстоянии между прямыми, между точкой и прямой, между точкой и окружностью и т.д. 

\textcolor{red}{\textbf{TODO: стоит подумать, нет ли ещё мест, где возникали проблемы с дифференцированием}}

\section{Выбор описания ограничений}
Ограничения описываются набором уравнений не единственным образом. Как минимум, потому, что если имеется несколько уравнений, описывающих данное ограничение, то все уравнения можно объединить в одно путём суммирования квадратов левых частей уравнений. Но уменьшение числа уравнений не обязательно приводит к лучшей сходимости решателя. Поэтому вариативность в способах описания ограничений даёт пространство для оптимизаций генерируемых уравнений, на уже существующей тестовой базе. Далее будут рассмотрены случаи, в которых было найдено более успешное описание ограничений, которое увеличило процент успешно решённых тестов на индустриальной тестовой базе. 

\textbf{Совпадение двух точек.} Данное ограничение можно трактовать как нулевое расстояние между точками и воспользоваться уравнениями \ref{eq:pt_pt_dist_good}. Но существует описание, при использовании которого процент успешно решенных тестов выше:
\begin{equation*}
    \left\{\begin{matrix}
    a_x - b_x & = 0\\
    a_y - b_y & = 0\\
    a_z - b_z & = 0
    \end{matrix}\right.
\end{equation*}
где: $\vec{a}, \vec{b}$ -- координаты двух точек. Этот пример, наглядно показывает, что минимизация числа уравнений в системе не лучший подход к выбору описания ограничений.

%Можно предположить, почему такое описание более удачно, воспользовавшись терминами степеней свободы объектов. Так, у двух свободных точек 6 степеней свободы. Однако ограничение совпадения снимает 3 степени свободы, т.к. положение одной из точек полностью определено положением другой. Соответственно // Крайне спекулятивный момент, который я е смогу ничем подтвердить, а значит писать это вряд-ли стоит

\textbf{Совпадение прямых.} Подобно совпадению двух точек, совпадение прямых можно описать б\'oльшим числом уравнений, чем расстояние между прямыми. Для этого введём обозначения: $\vec{d}$ -- направляющий вектор одной из прямых, $\vec{\Delta}$ -- вектор разности точек, лежащих на соответствующих прямых, $\vec{c} = \vec{\Delta}\times \vec{d} $. Тогда набор уравнений, удовлетворяющий ограничение совпадения, имеет вид:
\begin{equation*}
        c_x, c_y , c_z = 0
\end{equation*}
Помимо этого необходим набор уравнений обеспечивающий коллинеарность направляющих векторов прямых.

\textbf{Параллельность направляющих векторов.} Многие ограничения (совпадение, расстояние, касание и параллельность) требуют коллинеарности направляющих векторов. И на практике, в большинстве случаев оказывается более эффективным набор уравнений, по сравнению с одним уравнением, использующим скалярное произведение векторов. Также нужно учитывать параметр выравнивания, заданный в ограничении. Так при \textit{положительном} или \textit{отрицательном} выравнивании эффективным оказывается набор следующих уравнений:
\begin{equation*}
    \left\{\begin{matrix}
    a_x \pm b_x & = 0\\
    a_y \pm b_y & = 0\\
    a_z \pm b_z & = 0
    \end{matrix}\right.
\end{equation*}
где: $\vec{a}, \vec{b}$ -- направляющие вектора, знаку "$+$" соответствует случай отрицательного, а знаку "$-$" положительного выравнивания. В случае \textit{неопределённого} выравнивания эффективным оказывается приравнивание к нулю компоненты векторного произведения $\vec{c} = \vec{a} \times \vec{b}$:
\begin{equation*}
    c_x, c_y, c_z = 0
\end{equation*}

\textbf{Расстояние между параллельными прямыми.}

\textcolor{red}{\textbf{TODO: дописать секцию!}}

\textbf{Возможно стоит ещё описать работу со вспомогательными точками}